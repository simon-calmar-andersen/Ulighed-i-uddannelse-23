%% Præsentationsskabelon
%% Ulighed i Uddannelse BA-seminar
%% Simon Calmar Andersen


%%%%%%%%%%%%%%%%%%%%%%%%%%%%%%%%%%%%%%%%%%%%%%%%%%
%Opsætning
\documentclass{beamer}
\usepackage[utf8]{inputenc}

\usepackage[english]{babel}
\usepackage{amsmath,blkarray}
\usepackage{multirow}
\usepackage{outlines}
\usepackage{dsfont}
\usepackage{csquotes}
\usepackage[flushleft]{threeparttablex}


\mode<presentation> {

%\renewcommand{\labelitemi}{\footnotesize$\hookrightarrow$}
%\renewcommand{\labelitemii}{\footnotesize$\rightharpoondown$}


	%\usetheme{default}
	%\usetheme{AnnArbor}
	%\usetheme{Antibes}
	%\usetheme{Bergen}
	%\usetheme{Berkeley}
	%\usetheme{Berlin}
	%\usetheme{Boadilla}
	%\usetheme{CambridgeUS}
	%\usetheme{Copenhagen}
	%\usetheme{Darmstadt}
	%\usetheme{Dresden}
	%\usetheme{Frankfurt}
	%\usetheme{Goettingen}
	%\usetheme{Hannover}
	%\usetheme{Ilmenau}
	%\usetheme{JuanLesPins}
	%\usetheme{Luebeck}
	\usetheme{Madrid}
	%\usetheme{Malmoe}
	%\usetheme{Marburg}
	%\usetheme{Montpellier}
	%\usetheme{PaloAlto}
	%\usetheme{Pittsburgh}
	%\usetheme{Rochester}
	%\usetheme{Singapore}
	%\usetheme{Szeged}
	%\usetheme{Warsaw}
	
	% As well as themes, the Beamer class has a number of color themes
	% for any slide theme. Uncomment each of these in turn to see how it
	% changes the colors of your current slide theme.
	
	%\usecolortheme{albatross}
	%\usecolortheme{beaver}
	%\usecolortheme{beetle}
	%\usecolortheme{crane}
	%\usecolortheme{dolphin}
	%\usecolortheme{dove}
	%\usecolortheme{fly}
	\usecolortheme{lily}
	%\usecolortheme{orchid}
	%\usecolortheme{rose}
	%\usecolortheme{seagull}
	%\usecolortheme{seahorse}
	%\usecolortheme{whale}
	%\usecolortheme{wolverine}
	
	%\setbeamertemplate{footline} % To remove the footer line in all slides uncomment this line
	%\setbeamertemplate{footline}[page number] % To replace the footer line in all slides with a simple slide count uncomment this line
	
	%\setbeamertemplate{navigation symbols}{} % To remove the navigation symbols from the bottom of all slides uncomment this line
}

\useoutertheme{miniframes} %uncommet to remove header with sections

\usepackage{graphicx} % Allows including images
\usepackage{booktabs} % Allows the use of \toprule, \midrule and \bottomrule in tables
\usepackage{subfig}

\DeclareMathOperator\var{var}

\setbeamertemplate{footline}
{
  \leavevmode%
  \hbox{%
  \begin{beamercolorbox}[wd=.3\paperwidth,ht=2.25ex,dp=1ex,center]{author in head/foot}%
    \usebeamerfont{author in head/foot}\insertshortauthor
  \end{beamercolorbox}%
  \begin{beamercolorbox}[wd=.6\paperwidth,ht=2.25ex,dp=1ex,center]{title in head/foot}%
    \usebeamerfont{title in head/foot}\insertshorttitle
  \end{beamercolorbox}%
  \begin{beamercolorbox}[wd=.1\paperwidth,ht=2.25ex,dp=1ex,center]{date in head/foot}%
    \insertframenumber{} / \inserttotalframenumber\hspace*{1ex}
  \end{beamercolorbox}}%
  \vskip0pt%
}


%Outline
\AtBeginSection[]
{
\begin{frame}
\frametitle{Outline}
\tableofcontents[currentsection]
\end{frame}
}

% Referencer
\usepackage[
backend=bibtex,
style=authoryear,
doi=false,isbn=false,url=false,eprint=false
]{biblatex}
\addbibresource{bibliography.bib}

%----------------------------------------------------------------------------------------
%	TITLE PAGE
%----------------------------------------------------------------------------------------


\title{Effekten af sommerskole}

\author{Dit Navn}

\date{\today} %slet \today hvis ikke den skal opdatere dato

\institute[AU] % Your institution as it will appear on the bottom of every slide, may be shorthand to save space
{\large{Fiktivt datasæt}}

%------------------------------------
%	%	PRESENTATION SLIDES
%------------------------------------






%%%%%%%%%%%%%%%%%%%%%%%%%%%%%
% Document
%%%%%%%%%%%%%%%%%%%%%%%%%%%%%
\begin{document}
	
	\begin{frame}[plain]
		\titlepage % Print the title page as the first slide
	\end{frame}


%%%%%%%%%%%%%%%%%%%%%%%%%%
\section{Introduktion}
%%%%%%%%%%%%%%%%%%%%%%%%%%

\begin{frame}{Intro}
    \begin{itemize}
        \item Denne præsentation viser resultater af opgaver baseret på Sievertsens data og øvelser \href{https://hhsievertsen.github.io/applied_econ_with_r/}{(\underline{link})}. 
    \end{itemize}
\end{frame}







%%%%%%%%%%%%%%%%%%%%%%%%%%
\section{Data og design}
%%%%%%%%%%%%%%%%%%%%%%%%%%

\begin{frame}{Den empiriske case}
    \begin{itemize}
        \item Nogle elever blev tilfældigt udtrukket til at modtage en invitation til at deltage i sommerskole (indikeret med variablen Letter)
        \item Variablen Summer school indikerer de elever, som deltog på sommerskole
    \end{itemize}
\end{frame}



\begin{frame}{Balancetabel}
    \begin{table}[h] \centering
\resizebox{.7\linewidth}{!}{    	\begin{threeparttable} 
    		\caption{Balancetabel} \label{tab:balance}
    			\input{Sievertsen_ovelse/balance.tex}
    		\begin{tablenotes}
                \item Kilde: Data fra Sievertsen øvelse
            \end{tablenotes}
    	\end{threeparttable}}
    \end{table}
\end{frame}

%%%%%%%%%%%%%%%%%%%%%%%%%%
\section{Results}
%%%%%%%%%%%%%%%%%%%%%%%%%%


\begin{frame}{Compare means}

\begin{table}[h] \centering
	\begin{threeparttable} 
		\caption{Compare means, treatment} \label{tab:treatment}
			\input{Sievertsen_ovelse/treatment.tex}
		\begin{tablenotes}
            \item Kilde: Data fra Sievertsen øvelse
        \end{tablenotes}
	\end{threeparttable}
\end{table}

\end{frame}

\begin{frame}{Regression med kontrolvariable}

\begin{table}[h] \centering
\resizebox{.4\linewidth}{!}{
    \begin{threeparttable} 
		\caption{Sammenhæng mellem test score og summer school} \label{tab:regression}
			\input{Sievertsen_ovelse/regression.tex}
		\begin{tablenotes}
            \item Kilde: Data fra Sievertsen øvelse
        \end{tablenotes}
	\end{threeparttable}}
\end{table}
\end{frame}


\begin{frame}{Regression af randomiseret treatment (letter)}

\begin{table}[h] \centering
\resizebox{.4\linewidth}{!}{
    \begin{threeparttable} 
		\caption{Effekt af letter på test score} \label{tab:regression_letter}
			\input{Sievertsen_ovelse/regression_letter.tex}
		\begin{tablenotes}
            \item Kilde: Data fra Sievertsen øvelse
        \end{tablenotes}
	\end{threeparttable}}
\end{table}
\end{frame}

\begin{frame}{Instrumental Variable}


\begin{table}[h] \centering
\resizebox{.6\linewidth}{!}{
    \begin{threeparttable} 
		\caption{Reduced form og IV-model} \label{tab:iv}
			\input{Sievertsen_ovelse/IV_letter.tex}
		\begin{tablenotes}
            \item Kilde: Data fra Sievertsen øvelse
        \end{tablenotes}
	\end{threeparttable}}
\end{table}
\end{frame}

% \printbibliography

\end{document}

