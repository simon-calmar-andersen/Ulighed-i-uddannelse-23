%% Opgaveskabelon
%% Ulighed i Uddannelse BA-seminar
%% Simon Calmar Andersen


%%%%%%%%%%%%%%%%%%%%%%%%%%%%%%%%%%%%%%%%%%%%%%%%%%
%Opsætning


\documentclass[12pt]{article}
\usepackage[danish]{babel}
\usepackage[utf8]{inputenc}
\usepackage{graphicx} %For at kunne indsætte billedfiler såsom pdf fra f.eks. Stata-output
\usepackage{setspace} %For at kunne ændre linjeafstand
\usepackage{threeparttable} %For at kunne indsætte noter til tabel m.v.
\usepackage{booktabs} %For at kunne lave pænere tabel-layout. Fungerer sammen med booktabs-option i Stata

%For at kunne lave links til krydshenvisninger
\usepackage{hyperref} 
\hypersetup{
    colorlinks=true,
    allcolors=magenta
}

% Referencer
\usepackage[
backend=bibtex,
style=authoryear,
doi=false,isbn=false,url=false,eprint=false
]{biblatex}
\addbibresource{bibliography.bib}



%%%%%%%%%%%%%%%%%%%%%%%%%%%%%%%%%%%%%%%%%%%%%%%%%%
\title{Sievertsens opgaver 5: Balancetabel }

\author{Dit Navn$^{a*}$,  \\
        \small $^{a}$ Institut for Statskundskab, Aarhus Universitet \\
        \small $^{*}$ \tt{email.adresse}}

\date{\today} %slet \today hvis ikke den skal opdatere dato




%%%%%%%%%%%%%%%%%%%%%%%%%%%%%%%%%%%%%%%%%%%%%%%%%%%%%%
% Dokumentet begynder herunder
\begin{document}
\doublespacing

\maketitle

%\begin{abstract} 
%\noindent Her kan evt. skrives et kort abstact (sammenfatning) af opgaven.  
%\end{abstract}



\section*{Indledning}
Denne rapport viser resultater af opgave 5 i Sievertsens øvelser\footnote{\url{https://hhsievertsen.github.io/applied_econ_with_r/}}. 


\clearpage

\section*{Balancetabel}


Tabel \ref{tab:balance} viser balance ved baseline


\begin{table}[h] \centering
	\begin{threeparttable} 
		\caption{Balancetabel} \label{tab:balance}
			\input{Sievertsen_øvelse/balance.tex}
		\begin{tablenotes}
            \item Kilde: Data fra Sievertsen øvelse
        \end{tablenotes}
	\end{threeparttable}
\end{table}


\section*{Average Treatment Effect}

Tabel \ref{tab:treatment} gennemsnitlig, standardiseret test score for treatment (letter) og kontrolgruppe (no letter) - og om forskellen er signifikant.


\begin{table}[h] \centering
	\begin{threeparttable} 
		\caption{Compare means, treatment} \label{tab:treatment}
			\input{Sievertsen_øvelse/treatment.tex}
		\begin{tablenotes}
            \item Kilde: Data fra Sievertsen øvelse
        \end{tablenotes}
	\end{threeparttable}
\end{table}


% \printbibliography

\end{document}

