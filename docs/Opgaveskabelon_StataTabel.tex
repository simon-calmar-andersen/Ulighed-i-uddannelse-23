%% Opgaveskabelon
%% Ulighed i Uddannelse BA-seminar
%% Simon Calmar Andersen


%%%%%%%%%%%%%%%%%%%%%%%%%%%%%%%%%%%%%%%%%%%%%%%%%%
%Opsætning


\documentclass[12pt]{article}
\usepackage[danish]{babel}
\usepackage[utf8]{inputenc}
\usepackage{graphicx} %For at kunne indsætte billedfiler såsom pdf fra f.eks. Stata-output
\usepackage{setspace} %For at kunne ændre linjeafstand

%For at kunne lave links til krydshenvisninger
\usepackage{hyperref} 
\hypersetup{
    colorlinks=true,
    allcolors=magenta
}

% Referencer
\usepackage[
backend=bibtex,
style=authoryear,
doi=false,isbn=false,url=false,eprint=false
]{biblatex}
\addbibresource{bibliography.bib}



%%%%%%%%%%%%%%%%%%%%%%%%%%%%%%%%%%%%%%%%%%%%%%%%%%
\title{Titlen på din BA-opgave}

\author{Dit Navn$^{a*}$,  \\
        \small $^{a}$ Institut for Statskundskab, Aarhus Universitet \\
        \small $^{*}$ \tt{email.adresse}}

\date{\today} %slet \today hvis ikke den skal opdatere dato




%%%%%%%%%%%%%%%%%%%%%%%%%%%%%%%%%%%%%%%%%%%%%%%%%%%%%%
% Dokumentet begynder herunder
\begin{document}
\doublespacing

\maketitle

\begin{abstract} 
\noindent Her kan evt. skrives et kort abstact (sammenfatning) af opgaven.  
\end{abstract}



\section*{Indledning}
Beskriv den samfundsmæssige og forskningsmæssige motivation for din problemstilling. Indsæt evt en fodnote\footnote{Dette er en fodnote}. 


\section*{Referencer}
Lav en reference, til en artikel som du har eksporteret fra Zotero. Øverst i denne Latex-skabelon (linje 29 i skabelonen) skal du indsætte navnet på din bib-fil (som er den fil, der henter referencer fra Zotero)  og fjerne  \% -tegnet. I bib-bilen kan du se, hvilken "key" din reference har fået. Typisk er det \verb|efternavn_første-ord-titel_årstal|. Referencen indsættes med 

\verb|\parencite{andersen_increasing_2016}|.

\parencite{andersen_increasing_2016}
Nedest i Latex-dokumentet (lige oven ord \verb|\end{document}|) skal du fjerne \%-tegnet foran \verb|\printbibliography|.







\section*{Figurer}
Når du har uploaded en figur-fil til Overleaf-projektet, skal du indsætte dens navn herunder. Du kan henvise til figuren ved at indsætte reference til dens label \verb|\ref{figur-label}|. Fjern \%-tegnene fra tex-filen herunder, når figuren er uploaded og filnavnet ændret.

%\begin{figure}
%\centering
%\includegraphics[width=\linewidth]{navnet_på_din_fil.pdf}
%\caption{Eksempel på figur}
%\label{figur-label}
%\end{figure}


\section*{Tabeller}

Indsæt tabeller fra STata, som er gemt i undermappe.


\begin{table}
\input{eksempel/tabel.tex}
\end{table}

%\printbibliography

\end{document}

