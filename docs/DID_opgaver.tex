%% Opgaveskabelon
%% Ulighed i Uddannelse BA-seminar
%% Simon Calmar Andersen


%%%%%%%%%%%%%%%%%%%%%%%%%%%%%%%%%%%%%%%%%%%%%%%%%%
%Opsætning


\documentclass[12pt]{article}
\usepackage[danish]{babel}
\usepackage[utf8]{inputenc}
\usepackage{graphicx} %For at kunne indsætte billedfiler såsom pdf fra f.eks. Stata-output
\usepackage{setspace} %For at kunne ændre linjeafstand
\usepackage{threeparttable} %For at kunne indsætte noter til tabel m.v.

\usepackage{adjustbox} %For at kunne tilpasse tabel til siden
\usepackage{booktabs} %For at kunne lave pænere tabel-layout. Fungerer sammen med booktabs-option i Stata

%For at kunne lave links til krydshenvisninger
\usepackage{hyperref} 
\hypersetup{
    colorlinks=true,
    allcolors=magenta
}

% Referencer
\usepackage[
backend=bibtex,
style=authoryear,
doi=false,isbn=false,url=false,eprint=false
]{biblatex}
\addbibresource{bibliography.bib}



%%%%%%%%%%%%%%%%%%%%%%%%%%%%%%%%%%%%%%%%%%%%%%%%%%
\title{DiD opgaver }

\author{Simon Calmar Andersen$^{a*}$,  \\
        \small $^{a}$ Institut for Statskundskab, Aarhus Universitet \\
        \small $^{*}$ \tt{sca@ps.au.dk}}

\date{\today} %slet \today hvis ikke den skal opdatere dato




%%%%%%%%%%%%%%%%%%%%%%%%%%%%%%%%%%%%%%%%%%%%%%%%%%%%%%
% Dokumentet begynder herunder
\begin{document}
\doublespacing

\maketitle

%\begin{abstract} 
%\noindent Her kan evt. skrives et kort abstact (sammenfatning) af opgaven.  
%\end{abstract}



\section*{Indledning}
Denne rapport viser resultater af opgaver i Difference-in-Differences (DiD) metoden, baseret på Sievertsens data.\footnote{\url{https://hhsievertsen.github.io/applied_econ_with_r/}}


\clearpage

\section*{Common trend}




Figur \ref{common_trend} viser trend for elever der tog på summer school og elever, som ikke gjorde.


\begin{figure}
\caption{Common Trend Assumption} \label{common_trend}
\includegraphics{Sievertsen_ovelse/common_trend.pdf}
\end{figure}

\section*{DiD estimater}
Tabel \ref{tab:did} viser forskellige DiD modeller. Variablen \verb|camp_post| er en indikator (dummy) variabel med værdien 1 for observationer (elever), der har været på summcer camp i klassetrin efter de har været på summercamp (dvs. efter 5. klasse). Variablen \verb|post| er en indikator for klassetrin efter 5. klasse.

Elev-FE betyder elev-fixed-effects og svarer til at indsætte en dummy-variabel for hver elev i datasættet (svarende til Angrist og Pischke's ligning 5.5 s. 194, hvor de indsætter en dummy for hver stat i deres datasæt). Tilsvarende for klassetrin-FE.

\begin{table}[h] \centering
		\caption{DiD modeller med test scores som afhængig variabel} \label{tab:did}
	\begin{adjustbox}{max width=\textwidth}
	\begin{threeparttable} 
			\input{Sievertsen_ovelse/did.tex}
		\begin{tablenotes}
            \item Kilde: Data fra Sievertsen øvelse
        \end{tablenotes}
	\end{threeparttable}
	\end{adjustbox}
\end{table}

\section*{Opgaver}

\begin{enumerate}
	\item Forsøg at genskabe figur \ref{common_trend} og tabel \ref{tab:did}.
	\begin{enumerate}
	\item Stata-tip I: Figuren kan laves ved først at bruge kommandoen \verb|collapse| til at lave et datasæt, der tager gennemsnits-test-score for hver klassetrin (year) og for hver af de to grupper (summercamp eller ikke-summercamp)
	\item Stata-tip II: Regressionsmodellerne med fixed effects kan laves ved at installere pakken reghdfe (med kommandoerne \verb| ssc install ftools| \\ 
	og \verb|ssc install reghdfe|). Variabler, som skal bruges til fixed effect, sættes i absorb-parentesen i en kommando i stil med denne: \\ 
	\verb|reghdfe y x z, absorb(Fe_variabel) cluster(school_id)|
	\end{enumerate}
	\item Kommenter på figur \ref{common_trend}. Bestyrker den antagelsen om \textit{common trends}?
	\item Forklar tabel \ref{tab:did}. 
		\begin{enumerate}
		\item Hvad viser modellerne?		
		\item Hvorfor estimeres kontrolvariablene ikke i model 3-5 (selvom de var med i modellen i Stata)?
		\item Hvorfor er \verb|post| variablen ikke estimeret i model 4 og 5 (selvom den var med i modellerne)?
		\end{enumerate}
	\item Giv en samlet vurdering af om der er en effekt af summer camp (i dette fiktive datasæt) og hvor stor den er. Inddrag også gerne resultaterne fra regressionsmodellerne, RCT- og IV-modellerne fra de tidligere øvelsesopgaver.
\end{enumerate}

% \printbibliography

\end{document}

